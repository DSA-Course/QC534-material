
\documentclass[a4paper, 11pt, oneside]{book} % A4 paper size, default 11pt font size and oneside for equal margins
\usepackage{listings}
\newcommand{\plogo}{\fbox{$\mathcal{PL}$}} % Generic dummy publisher logo

\usepackage[utf8]{inputenc} % Required for inputting international characters
\usepackage[T1]{fontenc} % Output font encoding for international characters
\usepackage{fouriernc} % Use the New Century Schoolbook font

%----------------------------------------------------------------------------------------
%	TITLE PAGE
%----------------------------------------------------------------------------------------

\begin{document} 

\begin{titlepage} % Suppresses headers and footers on the title page

	\centering % Centre everything on the title page
	
	\scshape % Use small caps for all text on the title page
	
	\vspace*{\baselineskip} % White space at the top of the page
	
	%------------------------------------------------
	%	Title
	%------------------------------------------------
	
	\rule{\textwidth}{1.6pt}\vspace*{-\baselineskip}\vspace*{2pt} % Thick horizontal rule
	\rule{\textwidth}{0.4pt} % Thin horizontal rule
	
	\vspace{0.75\baselineskip} % Whitespace above the title
	
	{\LARGE Data structures\\ and\\ Algorithms\\} % Title
	
	\vspace{0.75\baselineskip} % Whitespace below the title
	
	\rule{\textwidth}{0.4pt}\vspace*{-\baselineskip}\vspace{3.2pt} % Thin horizontal rule
	\rule{\textwidth}{1.6pt} % Thick horizontal rule
	
	\vspace{2\baselineskip} % Whitespace after the title block
	
	%------------------------------------------------
	%	Subtitle
	%------------------------------------------------
	
	Note05-Stack, Queue and Deque % Subtitle or further description
	
	\vspace*{3\baselineskip} % Whitespace under the subtitle
	
	%------------------------------------------------
	%	Editor(s)
	%------------------------------------------------
	
	Edited By
	
	\vspace{0.5\baselineskip} % Whitespace before the editors
	
	{\scshape\Large Arshia Gharooni \\} % Editor list
	
	\vspace{0.5\baselineskip} % Whitespace below the editor list
	
	\vfill % Whitespace between editor names and publisher logo
	
	%------------------------------------------------
	%	Publisher
	%------------------------------------------------
	
	
	
	\vspace{0.3\baselineskip} % Whitespace under the publisher logo
	
	2023 % Publication year

\end{titlepage}

\section{Stack}
A stack is an abstract data type that represents a collection of elements, with two primary operations: push and pop. The push operation adds an element to the top of the stack, and the pop operation removes and returns the topmost element. Stacks are commonly used in computer science, particularly in algorithms and programming languages, for example, in parsing, recursion, and memory management.

\subsection{Implementation of stack}
A stack can be implemented using different data structures. The most common data structures used to implement a stack are an array and a linked list.

\subsubsection{Array implementation}  
                \\
   In an array implementation of a stack, we create an array of fixed size, and then we use two variables, namely top and size, to keep track of the elements in the stack. The top variable represents the index of the topmost element in the stack, and the size variable represents the number of elements in the stack.
                \\
                \\
   When we push an element onto the stack, we increment the top variable and insert the element at the top index of the array. When we pop an element from the stack, we remove the element at the top index of the array and decrement the top variable.
                \\
                \\
     The time complexity of push and pop operations in an array implementation of a stack is O(1), which means that the time it takes to perform these operations does not depend on the number of elements in the stack. However, the space complexity of an array implementation of a stack is O(n), where n is the size of the array. This means that we need to allocate a fixed amount of memory for the stack, regardless of the number of elements in the stack.
                \\
                \\
    Let's consider a simple example of implementing a stack using an array. Suppose we have an array of size 10, and we want to create a stack of integers using this array. We can use the following steps to implement the stack:
                
                
  \begin{enumerate}
       \item Initialize the stack: We start by initializing an integer variable called top to -1. This variable will be used to keep track of the topmost element in the stack.
                    
        \item Push an element: To push an element onto the stack, we first check if the stack is full or not. If the stack is not full, we increment the top variable and insert the new element at the top of the stack. If the stack is full, we display an error message saying that the stack is full.
                    
       \item Pop an element: To pop an element from the stack, we first check if the stack is empty or not. If the stack is not empty, we remove the topmost element from the stack and decrement the top variable. If the stack is empty, we display an error message saying that the stack is empty.
                    
        \item Peek the top element: To peek at the top element of the stack, we simply return the element at the index pointed to by the top variable.
\end{enumerate}


                \subsubsection{Linked List Implementation}
                In a linked list implementation of a stack, we use a linked list data structure to represent the stack. The linked list consists of nodes, with each node containing an element and a pointer to the next node in the list.
                \\
                \\
                To implement a stack using a linked list, we use a pointer variable, called top, to point to the topmost node in the linked list. When we push an element onto the stack, we create a new node and insert it at the beginning of the linked list, making it the new top node. When we pop an element from the stack, we remove the top node from the linked list and update the top variable to point to the next node in the list.
                \\
                \\
                The time complexity of push and pop operations in a linked list implementation of a stack is also O(1), which means that the time it takes to perform these operations does not depend on the number of elements in the stack. However, the space complexity of a linked list implementation of a stack is O(n), where n is the number of elements in the stack. This means that we need to allocate memory dynamically as we add elements to the stack.



\subsection{Stack Operations}
\subsubsection{Push}
Push operation is used to insert an element at the top of the stack. In this operation, we first check if the stack is full or not. If the stack is not full, we increment the top index and insert the element at the top index. The time complexity of push operation is O(1) as it involves only one operation.
\\
\\
The push operation involves only one operation, so the time complexity of push operation is O(1).
\subsubsection{Pop}
Pop operation is used to remove an element from the top of the stack. In this operation, we first check if the stack is empty or not. If the stack is not empty, we remove the element at the top index and decrement the top index. The time complexity of pop operation is also O(1) as it involves only one operation.
\\
\\
The pop operation involves only one operation, so the time complexity of pop operation is also O(1)
\subsubsection{Peek}
Peek operation is used to return the element at the top of the stack without removing it. In this operation, we first check if the stack is empty or not. If the stack is not empty, we return the element at the top index. The time complexity of peek operation is also O(1) as it involves only one operation.
\\
\\
The peek operation involves only one operation, so the time complexity of peek operation is also O(1).

\subsubsection{Size}
Size operation is used to return the number of elements in the stack. In this operation, we simply return the value of the top index plus one. The time complexity of size operation is also O(1) as it involves only one operation.
\\
\\
The size operation involves only one operation, so the time complexity of size operation is also O(1).




\section{Queue}
In computer science, a queue is an abstract data type that represents a collection of elements, where the addition of new elements happens at one end, and the removal of elements happens at the other end. It follows the principle of first-in, first-out (FIFO), meaning that the element that has been in the queue for the longest time is the first one to be removed. Queues are commonly used in programming to manage resources or to simulate real-world scenarios such as a line at a supermarket.
\\

In this lecture, we will discuss the various implementations of a queue data structure and the complexity of each implementation. We will also look at some common operations performed on queues and their corresponding time complexities.

\subsection{Implementation of Queue}
There are several ways to implement a queue data structure. The choice of implementation depends on the specific needs of the application and the constraints of the system. Here are some common implementations of a queue:

\begin{enumerate}
    \item Array-based Queue:
    An array-based queue is implemented using an array data structure. The front and rear of the queue are represented by two indices, and new elements are added to the rear of the queue and removed from the front of the queue.
    \\
    \\
    The time complexity of inserting an element into the queue is O(1) since we can simply append the element to the end of the array. The time complexity of removing an element from the queue is also O(1) since we can simply remove the element at the front of the array. However, in the worst case scenario when the array becomes full, we need to resize the array which takes O(n) time where n is the size of the array.
    \item Linked List-based Queue:
    A linked list-based queue is implemented using a linked list data structure. The front and rear of the queue are represented by two pointers, and new elements are added to the rear of the queue and removed from the front of the queue.  
    \\
    \\
    The time complexity of inserting an element into the queue is O(1) since we can simply add a new node to the end of the linked list. The time complexity of removing an element from the queue is also O(1) since we can simply remove the first node from the linked list.
    \item Circular Queue:
    A circular queue is implemented using an array data structure with a fixed size. The front and rear of the queue are represented by two indices that wrap around to the beginning of the array when they reach the end. New elements are added to the rear of the queue and removed from the front of the queue.
    \\
    \\
    The time complexity of inserting an element into the queue is O(1) since we can simply append the element to the end of the array. The time complexity of removing an element from the queue is also O(1) since we can simply remove the element at the front of the array. However, in the worst case scenario when the array becomes full, we cannot insert any more elements even if there are empty spaces in the array.            
\end{enumerate}

\subsection{Queue Operations}
A queue supports two basic operations, enqueue, and dequeue.

\subsubsection{Enqueue}
The enqueue operation adds an item to the end of the queue. The new item becomes the last item in the queue.

\begin{itemize}
    \item Algorithm:
    \\
    \begin{lstlisting}
        def  Enqueue(Queue, item)
            add item to the end of the Queue
    \end{lstlisting}    
    \item Time complexity:
    \\
    The time complexity of the enqueue operation is O(1) because it involves only one operation, which is adding an item to the end of the queue.
\end{itemize}

\subsubsection{Dequeue}
The dequeue operation removes the item at the front of the queue. The item is removed from the queue, and the next item in the queue becomes the front item.

\begin{itemize}
    \item Algorithm:
    \\
    \begin{lstlisting}
        def Dequeue(Queue)
            if Queue is not empty
                remove the first item from the Queue
    \end{lstlisting}    
    \item Time complexity:
    \\
    The time complexity of the dequeue operation is also O(1) because it involves only one operation, which is removing the first item from the queue.
\end{itemize}

\subsubsection{Additional Queue Operations}
In addition to the basic operations, a queue can also support other operations, such as checking if the queue is empty and finding the size of the queue.
\begin{enumerate}
    \item IsEmpty:
    \\
    The IsEmpty operation checks if the queue is empty. If the queue is empty, it returns true; otherwise, it returns false.
    \begin{itemize}
    \item Algorithm:
    \\
    \begin{lstlisting}
        def IsEmpty(Queue)
            if Queue is empty
                return true
            else
                return false
    \end{lstlisting}    
    \item Time complexity:
    \\
    The time complexity of the IsEmpty operation is O(1) because it involves only one operation, which is checking if the queue is empty.
    \end{itemize}

    \item Size:
    \\
    The Size operation returns the number of items in the queue.
    \begin{itemize}
    \item Algorithm:
    \\
    \begin{lstlisting}
        def Size(Queue)
            return the number of items in the Queue
    \end{lstlisting}    
    \item Time complexity:
    \\
    The time complexity of the Size operation is O(1) because it involves only one operation, which is returning the number of items in the queue.
    \end{itemize}
\end{enumerate}

\section{Deque}
A deque, short for double-ended queue, is a data structure that allows insertion and removal of elements from both ends. In other words, it is a linear collection of elements that supports adding and removing elements from both the front and the back of the queue. The deque data structure is useful in many situations, including implementing algorithms for breadth-first search, simulation of queueing systems, and for implementing data structures such as stacks and queues.

\subsection{Deque Implementations}
There are several ways to implement a deque data structure, each with its own advantages and disadvantages. Here are some of the most common implementations:

\subsubsection{Array-based Deque}
An Array-based Deque is a data structure that is used to store a collection of elements in a particular order. The term Deque is an abbreviation for Double Ended Queue, and it refers to a data structure that allows insertion and deletion of elements from both ends of the queue. In this lecture note, we will discuss the implementation and usage of Array-based Deque.

\begin{itemize}
    \item Implementation of Array-based Deque
    \\
    Array-based Deque is implemented using an array data structure. In this implementation, the array is used to store the elements of the Deque. The two ends of the Deque are represented by two indices, front and rear, which point to the first and last elements of the Deque, respectively. Initially, both front and rear are set to -1, which indicates that the Deque is empty.

    \item Operations on Array-based Deque
    \\
    The following operations can be performed on an Array-based Deque.
    \begin{enumerate}
        \item Insertion:
        \\
        Elements can be inserted into an Array-based Deque from either end. When an element is inserted into the Deque, the front and rear indices are updated accordingly. If the Deque is empty, then both front and rear are set to 0. If the Deque is full, then no more elements can be inserted.
        \item Deletion:
        \\
        Elements can be deleted from an Array-based Deque from either end. When an element is deleted from the Deque, the front and rear indices are updated accordingly. If the Deque becomes empty after deletion, then both front and rear are set to -1.
        \item  Peek:
        \\
        The Peek operation is used to retrieve the element at either end of the Deque without removing it. This operation does not modify the Deque in any way.
        
        \item Size:
        \\
        The Size operation is used to determine the number of elements in the Deque. 
    \end{enumerate}
\end{itemize}


\subsubsection{Linked-list-based Deque}
A Linked-list-based Deque is a Deque implementation using a linked list as the underlying data structure. A Linked-list-based Deque has two pointers: head and tail. The head pointer points to the first node of the list, while the tail pointer points to the last node of the list.

\begin{itemize}
    \item Insertion
    \\
    Insertion at the head or the tail of the deque can be done by simply adding a new node to the linked list and changing the pointers accordingly. To insert at the head, we create a new node and make it the new head by changing the next pointer of the new node to the current head and updating the head pointer to point to the new node. To insert at the tail, we create a new node and make it the new tail by changing the next pointer of the current tail to the new node and updating the tail pointer to point to the new node. In both cases, the time complexity is O(1).

    \item Removal
    \\
    Removal from the head or the tail of the deque can be done by simply removing the corresponding node from the linked list and changing the pointers accordingly. To remove from the head, we remove the current head node and update the head pointer to point to the next node. To remove from the tail, we remove the current tail node and update the tail pointer to point to the previous node. In both cases, the time complexity is O(1).
    
\end{itemize}

\end{document}