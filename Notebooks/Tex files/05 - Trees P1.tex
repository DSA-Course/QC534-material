
\documentclass[a4paper, 11pt, oneside]{book} % A4 paper size, default 11pt font size and oneside for equal margins

\newcommand{\plogo}{\fbox{$\mathcal{PL}$}} % Generic dummy publisher logo

\usepackage[utf8]{inputenc} % Required for inputting international characters
\usepackage[T1]{fontenc} % Output font encoding for international characters
\usepackage{fouriernc} % Use the New Century Schoolbook font
\usepackage{listings}
%----------------------------------------------------------------------------------------
%	TITLE PAGE
%----------------------------------------------------------------------------------------

\begin{document} 

\begin{titlepage} % Suppresses headers and footers on the title page

	\centering % Centre everything on the title page
	
	\scshape % Use small caps for all text on the title page
	
	\vspace*{\baselineskip} % White space at the top of the page
	
	%------------------------------------------------
	%	Title
	%------------------------------------------------
	
	\rule{\textwidth}{1.6pt}\vspace*{-\baselineskip}\vspace*{2pt} % Thick horizontal rule
	\rule{\textwidth}{0.4pt} % Thin horizontal rule
	
	\vspace{0.75\baselineskip} % Whitespace above the title
	
	{\LARGE Data structures\\ and\\ Algorithms\\} % Title
	
	\vspace{0.75\baselineskip} % Whitespace below the title
	
	\rule{\textwidth}{0.4pt}\vspace*{-\baselineskip}\vspace{3.2pt} % Thin horizontal rule
	\rule{\textwidth}{1.6pt} % Thick horizontal rule
	
	\vspace{2\baselineskip} % Whitespace after the title block
	
	%------------------------------------------------
	%	Subtitle
	%------------------------------------------------
	
	Note06-Introduction to trees part01% Subtitle or further description
	
	\vspace*{3\baselineskip} % Whitespace under the subtitle
	
	%------------------------------------------------
	%	Editor(s)
	%------------------------------------------------
	
	Edited By
	
	\vspace{0.5\baselineskip} % Whitespace before the editors
	
	{\scshape\Large Arshia Gharooni \\} % Editor list
	
	\vspace{0.5\baselineskip} % Whitespace below the editor list
	
	\vfill % Whitespace between editor names and publisher logo
	
	%------------------------------------------------
	%	Publisher
	%------------------------------------------------
	
	
	
	\vspace{0.3\baselineskip} % Whitespace under the publisher logo
	
	2023 % Publication year

\end{titlepage}

\section{Introduction}
Trees are a fundamental data structure in computer science and are used to represent hierarchical relationships between data. Trees have applications in many areas of computer science, including algorithms, databases, graphics, and artificial intelligence.
\\
\\
A tree is a data structure that consists of nodes connected by edges. Each node in a tree can have zero or more child nodes, except for the root node, which has no parent node. The edges in a tree represent the relationships between the nodes. Trees are used to represent hierarchical relationships between data.

\section{Some importation definitions}
\subsection{Subtree}
A subtree is a subset of a tree that can be separated and treated as its own smaller tree. It consists of a node, called the root of the subtree, and all of its descendants. Formally, let T be a tree with root r, then the subtree rooted at node u is the tree $T_u = (V_u, E_u)$ where $V_u$ is the set of nodes that are descendants of u in T, and $E_u$ is the set of edges that connect these nodes.

\subsection{Depth of a Node}
The depth of a node u in a tree T is denoted by $d_T(u)$ and defined as the length of the path from the root of the tree to u. The length of a path is the number of edges on the path.
\\
\\
Trivially, we can define the depth of a node u as follows:
\begin{itemize}
    \item $d_T(u)$ = 0, if u is the root of T.
    \item $d_T(u)$ = $d_T(v)$ + 1, if u is a child of v in T.
\end{itemize}


\subsection{Height of a Node}
The height of a node u in a tree T is denoted by $h_T(u)$ and defined as the length of the longest path from u to a leaf node in T.
\\
\\
Trivially, we can define the height of a node u as follows:
\begin{itemize}
    \item $h_T(u)$ = 0, if u is a leaf node.
    \item $h_T(u)$ = max$(h_T(v))$ + 1, if u is not a leaf node and has children $v_1$, $v_2$,..., $v_k$ in T.
\end{itemize}


\section{Binary trees}
 A binary tree is a tree data structure where each node has at most two children, referred to as the left child and the right child. The left child is usually smaller than the right child in value. The nodes are connected by edges or pointers. The topmost node in a binary tree is called the root, while the nodes without children are called leaf nodes.

\section{Tree Traversals}
Tree traversal is a fundamental operation on trees that is used in many algorithms, including searching, sorting, and indexing. In this lecture note, we will discuss the three main types of tree traversal: pre-order, in-order, and post-order. 

\subsection{Tree Traversal Types}
\subsubsection{Pre-order Traversal}

Pre-order traversal is a method of visiting each node in a tree in a specific order. In pre-order traversal, we visit the root node first, followed by the left subtree, and then the right subtree. In other words, we process the node before its children.

\subsubsection{In-order Traversal(Traversal order)}
In-order traversal is another method of visiting each node in a tree in a specific order. In in-order traversal, we visit the left subtree first, followed by the root node, and then the right subtree. In other words, we process the node between its children. 

\subsubsection{Post-order Traversal}
Post-order traversal is the third and final method of visiting each node in a tree in a specific order. In post-order traversal, we visit the left subtree first, followed by the right subtree, and then the root node. In other words, we process the node after its children.
\\


\section{Tree Navigation}
Tree navigation is a fundamental operation when we are talking about trees and their implementation.

\subsection{Finding the First Node in the Traversal Order}
Given a node <X>, we want to find the first node in its subtree according to some traversal order so the whole algorithm will be like:
\begin{enumerate}
    \item  If <X> has left child, recursively return the first node in the left subtree
    \item  Otherwise, <X> is the first node, so return it
\end{enumerate}
\\
\\
The time complexity of this algorithm is O(h), where h is the height of the tree. This is because in the worst case, we may have to traverse the entire height of the tree to find the first node. In the best case, the first node may be the root node, and we can return it in constant time.

\subsection{Finding the Successor of a Node}
Given a node <X>, we want to find its successor in the traversal order, which is the next node we would visit if we were to continue the traversal after visiting <X>. To find the successor, we use the following algorithm:

\begin{enumerate}
    \item If <X> has a right child, return the first node in the right subtree.
    \item Otherwise, return the lowest ancestor of <X> for which <X> is in its left subtree.
\end{enumerate}
\\

The time complexity of this algorithm is also O(h), where h is the height of the tree. This is because, in the worst case, we may have to traverse the entire height of the tree to find the successor. In the best case, the successor may be the right child of <X>, and we can return it in constant time.


\section{Dynamic Operations on Binary Search Trees}
Binary Search Trees (BST) are a popular data structure that supports various dynamic operations such as insertion and deletion of nodes. This lecture note covers the dynamic operations of BST and their implementation.

\subsection{Insertion operation}
Suppose we want to insert a new node <Y> after an existing node <X> in the traversal order. We can perform the insertion operation as follows:
\begin{enumerate}
    \item If <X> has no right child, make <Y> the right child of <X>.
    \item Otherwise, make <Y> the left child of <X>'s successor (which cannot have a left child).
\end{enumerate}
\\
The running time of this insertion operation is O(h), where h is the height of the tree.


\subsection{Deletion operation}
Suppose we want to remove an item from the subtree rooted at a node <X>. We can perform the deletion operation as follows:

\begin{enumerate}
    \item If <X> is a leaf, detach it from the parent and return.
    \item Otherwise, <X> has a child.
    \begin{enumerate}
        \item If <X> has a left child, swap items with the predecessor of <X> and recurse.
        \item Otherwise, <X> has a right child, swap items with the successor of <X> and recurse.
    \end{enumerate}
\end{enumerate}
\\
The running time of this deletion operation is O(h).


\section{Binary search tree (BST)}
A Binary Search Tree (BST) is a fundamental data structure used in computer science for efficient searching, insertion, and deletion operations. It is a tree data structure in which each node has at most two children, known as left and right children. In addition, each node in a BST holds a key, which is used to order and search the tree. The key in each node is unique.
\\

A BST is a particular type of tree where every node in the left subtree of a given node holds a smaller key value, while every node in the right subtree holds a larger key value. This ordering of keys in a BST allows for efficient searching, as the search algorithm can traverse the tree by comparing the search key with the keys in each node and following the appropriate subtree.

\subsection{Properties of Binary Search Trees}
A BST has several essential properties that make it a useful data structure for searching and sorting:

\begin{enumerate}
    \item The key in each node is unique.
    \item The left and right subtrees of a node are also BSTs.
    \item All the keys in the left subtree of a node are less than the key in the node.
    \item All the keys in the right subtree of a node are greater than the key in the node.
\end{enumerate}
The first property is essential for the correct functioning of the BST. The second property ensures that we can apply the same search algorithm to the left and right subtrees, as they are also BSTs. The third and fourth properties provide the ordering of the keys in the tree, which enables efficient searching.


\section{BST Operations}
\subsection{Search/find}
The search operation is one of the most basic operations on a BST. It allows us to find a specific node in the tree by its value. The search operation can be implemented recursively or iteratively.


\begin{itemize}
    \item Recursive search implementation:
        \begin{lstlisting}
        def search(node, value):
            if node is None or node.value == value:
                return node
            if value < node.value:
                return search(node.left, value)
            else:
                return search(node.right, value)
        \end{lstlisting}
    \item Iterative search implementation:
        \begin{lstlisting}
        def search(node, value):
            while node is not None and node.value != value:
                if value < node.value:
                    node = node.left
                else:
                    node = node.right
            return node
        \end{lstlisting}
\end{itemize}

\subsection{Insertion}
The insertion operation allows us to add a new node to the tree while maintaining the BST property. The insertion operation can be implemented recursively or iteratively.

\begin{itemize}
    \item Recursive insertion implementation
        \begin{lstlisting}
        def insert(node, value):
            if node is None:
                return Node(value)
            if value < node.value:
                node.left = insert(node.left, value)
            else:
                node.right = insert(node.right, value)
            return node
        \end{lstlisting}
    \item Iterative insertion implementation:
        \begin{lstlisting}
        def insert(node, value):
            if node is None:
                return Node(value)
            parent = None
            current = node
            while current is not None:
                parent = current
                if value < current.value:
                    current = current.left
                else:
                    current = current.right
            new_node = Node(value)
            if value < parent.value:
                parent.left = new_node
            else:
                parent.right = new_node
            return node
        \end{lstlisting}

\end{itemize}



\subsection{Deletion}
The deletion operation in BSTs removes a node with a given value from the tree while maintaining the BST property. There are three cases to consider:

\begin{enumerate}
    \item The node to be deleted is a leaf node (has no children).
    \item The node to be deleted has one child.
    \item The node to be deleted has two children.
\end{enumerate}

In the first case, we simply remove the node from the tree. In the second case, we replace the node with its child. In the third case, we replace the node with its successor (the smallest node in its right subtree), and recursively delete the successor.

\begin{itemize}
    \item Recursive deletion implementation
        \begin{lstlisting}
        def delete(node, value):
            if node is None:
                return node
            if value < node.value:
                node.left = delete(node.left, value)
            elif value > node.value:
                node.right = delete(node.right, value)
            else:
                if node.left is None:
                    return node.right
                elif node.right is None:
                    return node.left
                else:
                    successor = node.right
                    while successor.left is not None:
                        successor = successor.left
                    node.value = successor.value
                    node.right = delete(node.right, successor.value)
            return node
        \end{lstlisting}
    \item Iterative Deletion implementation:
        \begin{lstlisting}
        def delete(node, value):
            parent = None
            current = node
            while current is not None and current.value != value:
                parent = current
                if value < current.value:
                    current = current.left
                else:
                    current = current.right
            if current is None:
                return node
            if current.left is None:
                child = current.right
            elif current.right is None:
                child = current.left
            else:
                successor = current.right
                while successor.left is not None:
                    successor = successor.left
                current.value = successor.value
                current = successor
                child = current.right
            if parent is None:
                node = child
            elif parent.left == current:
                parent.left = child
            else:
                parent.right = child
            return node
        \end{lstlisting}

\end{itemize}


\end{document}