
\documentclass[a4paper, 11pt, oneside]{book} % A4 paper size, default 11pt font size and oneside for equal margins

\newcommand{\plogo}{\fbox{$\mathcal{PL}$}} % Generic dummy publisher logo

\usepackage[utf8]{inputenc} % Required for inputting international characters
\usepackage[T1]{fontenc} % Output font encoding for international characters
\usepackage{fouriernc} % Use the New Century Schoolbook font

%----------------------------------------------------------------------------------------
%	TITLE PAGE
%----------------------------------------------------------------------------------------

\begin{document} 

\begin{titlepage} % Suppresses headers and footers on the title page

	\centering % Centre everything on the title page
	
	\scshape % Use small caps for all text on the title page
	
	\vspace*{\baselineskip} % White space at the top of the page
	
	%------------------------------------------------
	%	Title
	%------------------------------------------------
	
	\rule{\textwidth}{1.6pt}\vspace*{-\baselineskip}\vspace*{2pt} % Thick horizontal rule
	\rule{\textwidth}{0.4pt} % Thin horizontal rule
	
	\vspace{0.75\baselineskip} % Whitespace above the title
	
	{\LARGE Data structures\\ and\\ Algorithms\\} % Title
	
	\vspace{0.75\baselineskip} % Whitespace below the title
	
	\rule{\textwidth}{0.4pt}\vspace*{-\baselineskip}\vspace{3.2pt} % Thin horizontal rule
	\rule{\textwidth}{1.6pt} % Thick horizontal rule
	
	\vspace{2\baselineskip} % Whitespace after the title block
	
	%------------------------------------------------
	%	Subtitle
	%------------------------------------------------
	
	Note04-Data Structures % Subtitle or further description
	
	\vspace*{3\baselineskip} % Whitespace under the subtitle
	
	%------------------------------------------------
	%	Editor(s)
	%------------------------------------------------
	
	Edited By
	
	\vspace{0.5\baselineskip} % Whitespace before the editors
	
	{\scshape\Large Arshia Gharooni \\} % Editor list
	
	\vspace{0.5\baselineskip} % Whitespace below the editor list
	
	\vfill % Whitespace between editor names and publisher logo
	
	%------------------------------------------------
	%	Publisher
	%------------------------------------------------
	
	
	
	\vspace{0.3\baselineskip} % Whitespace under the publisher logo
	
	2023 % Publication year

\end{titlepage}

\section{Data structures}
Data structures are a fundamental concept in computer science that is essential for efficient programming. In simple terms, a data structure is a way to organize and store data, with algorithms that support operations on the data. The collection of supported operations is called an interface, also known as an API or ADT. In this note, we will discuss two main interfaces of data structures, which are Sequence and Set. We will also discuss the special case interfaces of stack, queue, and dictionary.

\section{Data Structure Interfaces}
A data structure interface is a specification of the operations that a data structure should support. The interface defines what operations are supported and what parameters and data types are expected. On the other hand, a data structure is a representation of the actual implementation of those operations. The data structure specifies how the operations are supported and how the data is stored.

\subsection{Basic Functionality}
The basic functionality of a data structure refers to the operations it can perform, such as adding, removing, searching, and retrieving elements. The functionality of a data structure is defined by its interface, which includes its methods and properties. For example, the basic functionality of an array includes adding and removing elements, accessing elements by index, and determining the length of the array.

\subsection{Properties}
Properties are the attributes of a data structure that describe its state. They can be used to determine the size, capacity, and other characteristics of the data structure. Examples of properties include:
\begin{itemize}
    \item size – the number of elements in the data structure
    \item capacity – the maximum number of elements that can be stored in the data structure
    \item isEmpty – a Boolean value that indicates whether the data structure is empty or not
\end{itemize}

\subsection{Implementation of Interfaces}
The implementation of the interfaces of data structures is an important consideration when designing and developing software. There are several factors that can influence the implementation of data structure interfaces, including performance, memory usage, and ease of use.

\subsubsection{Performance}
Performance is a critical factor when implementing data structure interfaces. The performance of a data structure is determined by its efficiency in terms of time and space complexity. For example, an array has O(1) time complexity for accessing elements by index, but O(n) time complexity for inserting or deleting elements in the middle of the array.
\subsubsection{Memory Usage}
Memory usage is another important consideration when implementing data structure interfaces. The memory requirements of a data structure can affect the overall performance of a program, as well as its scalability. For example, a linked list uses more memory than an array, but it can be more efficient for inserting or deleting elements in the middle of the list.
\subsubsection{Ease of Use}
Ease of use is also a factor when implementing data structure interfaces. The interface of a data structure should be intuitive and easy to use, even for developers who are not familiar with the specific implementation. For example, the interface of an ArrayList in Java is similar to that of an array, making it easy for developers to switch between the two.

\subsection{Types of Interfaces}
There are two main interfaces of data structures, which are Sequence and Set. We will also discuss the special case interfaces of stack, queue, and dictionary.

\subsubsection{Set Interface}



\subsubsection{Sequence Interface}
The sequence interface maintains a sequence of items where order is extrinsic. This means that the order is explicitly specified and is not inherent in the data. An example of a sequence is (x0, x1, x2, …, xn-1), where zero indexing is used. The sequence interface supports the following sequence operations:

\begin{itemize}
    \item Container build(X): Given an iterable X, build a sequence from the items in X.
    \item len(): Return the number of stored items.
    \item Static iter seq(): Return the stored items one-by-one in sequence order.
    \item get at(i): Return the i-th item.
    \item set at(i, x): Replace the i-th item with x.
    \item Dynamic insert at(i, x): Add x as the i-th item.
    \item delete at(i): Remove and return the i-th item.
    \item insert first(x): Add x as the first item.
    \item delete first(): Remove and return the first item.
    \item insert last(x): Add x as the last item.
    \item delete last(): Remove and return the last item.
\end{itemize}

The sequence interface can be used to implement other data structures such as stacks and queues. A stack is a data structure that supports inserting and deleting elements from the top of the stack, while a queue supports inserting elements at the rear and deleting elements from the front.


\subsubsection{Set Interface}
The set interface maintains a set of items where order is intrinsic. This means that the order is inherent in the data and not explicitly specified. The set interface supports the following set operations:


\begin{itemize}
    \item Container build(X): Given an iterable X, build a set from the items in X.
    \item len(): Return the number of stored items.
    \item Static find(k): Return the stored item with key k.
    \item Dynamic insert(x): Add x to the set (replace item with key x.key if one already exists).
    \item delete(k): Remove and return the stored item with key k.
    \item Order iter ord(): Return the stored items one-by-one in key order.
    \item find min(): Return the stored item with the smallest key.
    \item find max(): Return the stored item with the largest key.
    \item find next(k): Return the stored item with the smallest key larger than k.
    \item find prev(k): Return the stored item with the largest key smaller than k.
\end{itemize}


The set interface can be used to implement a dictionary, which is a data structure that supports storing key-value pairs. A dictionary is similar to a set, but instead of storing only keys, it stores both keys and their associated values.
\end{document}

